%!TEX program = lualatex

\documentclass[17pt]{beamer} % notes, notes=only
%\documentclass[handout]{beamer}
%\documentclass[aspectratio=169]{beamer}

\usepackage{fontspec}
\usepackage{libertine}
%\usepackage{microtype}
\usepackage{amssymb}
\usepackage{mathtools}
\usepackage{unicode-math}
\usepackage{lualatex-math}

\setmainfont[Ligatures=TeX]{Linux Biolinum O}
%\setmathfont[math-style=ISO,bold-style=ISO,vargreek-shape=TeX,Ligatures=TeX]{TeX Gyre Pagella Math}

\usepackage{polyglossia} % babel replacement for use with fontspec
\setdefaultlanguage[variant=american]{english}
\selectlanguage[variant=american]{english}
\usepackage{csquotes}
\usepackage[load-configurations={abbreviations,binary}]{siunitx}
\sisetup{load-configurations = abbreviations,binary-units}
\usepackage{multirow}
\usepackage{tabu}
\usepackage{booktabs}
\usepackage{algorithmic}
\usepackage{algorithm}
\usepackage{glossaries}
\usepackage{color}

\usetheme{EastLansing}
\usecolortheme{seagull}
\useinnertheme{rectangles}
\usefonttheme{professionalfonts}

\beamertemplatenavigationsymbolsempty
\AtBeginSection{\sectionpage}


\title[]{Generalversammlung FunkFeuer Wien}
%\author{}
%\institute[]{}
\date[]{2019-05-27\\\vspace{1cm}\tiny CC BY-SA 4.0}


\def\Put(#1,#2)#3{\leavevmode\makebox(0,0){\put(#1,#2){#3}}}

\newcommand\scaleheight{0.7}



\begin{document}


\frame{\titlepage}

%%%%%%%%%%%%%%%%%%%%%%%%%%%%%%%%%%%%%%%%%%%%%%%%%%%%%%%%%%%%%%%%%%%%%%%%
%
% \begin{frame}
% 	\frametitle{Outline}
% 	\tableofcontents
% \end{frame}
% \note[itemize] {
% 	\item NOTIZ HIER
% }

%%%%%%%%%%%%%%%%%%%%%%%%%%%%%%%%%%%%%%%%%%%%%%%%%%%%%%%%%%%%%%%%%%%%%%%%
\begin{frame}
	\frametitle{Begrüßung}
	Beginn um 18:30.
	Gäste-WLAN \textit{TBD}
\end{frame}



%%%%%%%%%%%%%%%%%%%%%%%%%%%%%%%%%%%%%%%%%%%%%%%%%%%%%%%%%%%%%%%%%%%%%%%%
\begin{frame}
	\frametitle{Feststellen der Beschlussfähigkeit}
	nach §9 Abs. 6:
	\textit{,,Die Generalversammlung ist ohne Rücksicht auf die Anzahl der
	Erschienenen beschlussfähig.''}
\end{frame}



%%%%%%%%%%%%%%%%%%%%%%%%%%%%%%%%%%%%%%%%%%%%%%%%%%%%%%%%%%%%%%%%%%%%%%%%
\begin{frame}
	\frametitle{Was haben wir vor uns?}
	\begin{itemize}
		\item Berichte von Vorstand, Arbeitsgruppen, Kassieren,
		Rechnungsprüfern
		\item Abstimmung über Entlastung
		\item Vorstandswahl
		\item Abstimmung über Anträge zu den Statuten und zur Tagesordnung
		\item Allfälliges --- ,,Neuhousing''
	\end{itemize}
\end{frame}



%%%%%%%%%%%%%%%%%%%%%%%%%%%%%%%%%%%%%%%%%%%%%%%%%%%%%%%%%%%%%%%%%%%%%%%%
\begin{frame}
	\frametitle{Bericht des Vorstands}
	...inklusive Berichte aus Arbeitsgruppen
	\begin{itemize}
		\item Anexia-Roofnode; Backbone
		\item 60-GHz-Strecke NIX--Krypta
		\item Infrastruktur: VM-Übersiedlung
		\item Housing-Umzug (später mehr)
		\item MoMos, FMB, Kommunikation, verfehlte Ziele
	\end{itemize}
\end{frame}
\note[itemize] {
	\item Zirka 2 Minuten pro Punkt
	\item Vortrag muss nicht durch den Vorstand erfolgen!
}



%%%%%%%%%%%%%%%%%%%%%%%%%%%%%%%%%%%%%%%%%%%%%%%%%%%%%%%%%%%%%%%%%%%%%%%%
\begin{frame}
	\frametitle{Bericht Anexia und Backbone-Team}
	Stefan SCHULTHEIS:
	\begin{itemize}
		\item Kooperation und Roofnode Anexia
		\item Statusupdate Backbone
	\end{itemize}
\end{frame}



%%%%%%%%%%%%%%%%%%%%%%%%%%%%%%%%%%%%%%%%%%%%%%%%%%%%%%%%%%%%%%%%%%%%%%%%
\begin{frame}
	\frametitle{Bericht 60-GHz-Strecke NIX--Krypta}
\end{frame}



%%%%%%%%%%%%%%%%%%%%%%%%%%%%%%%%%%%%%%%%%%%%%%%%%%%%%%%%%%%%%%%%%%%%%%%%
\begin{frame}
	\frametitle{Bericht MoMos, FMB, Kommunikation, verfehlte Ziele}
\end{frame}
\note[itemize] {
	\item MoMos sind sehr gut besucht (5-20 Leute, alt/jung/neu/erfahren). Danke David für die Initiative!
	\item FMB: Funküberwachung meldete einen auf Indoor-Kanal konfigurierten Knoten. Schnelles Handeln nötig, öffentliche Wirkung (auf Verein und Ministerium) gewünscht; privat negatives Feedback
	\item Kommunikation: schnell, diskret, umfassend $\rightarrow$ wähle zwei!
	\item Bitte so öffentlich wie möglich kommunizieren
	\item Alle sollen möglichst alles wissen können
	\item Bitte auch heraus mit Berichten über laufende Projekte! (New-Node-Monitoring, New-Link-Monitoring, ...)
	\item Verfehlte Ziele: Kontakt zu Maintainern. Öffentliche Vorstandsliste. FunkFeuer und ,,Kommerz''
}



%%%%%%%%%%%%%%%%%%%%%%%%%%%%%%%%%%%%%%%%%%%%%%%%%%%%%%%%%%%%%%%%%%%%%%%%
\begin{frame}
	\frametitle{Bericht 60-GHz-Strecke NIX--Krypta}
	
\end{frame}



%%%%%%%%%%%%%%%%%%%%%%%%%%%%%%%%%%%%%%%%%%%%%%%%%%%%%%%%%%%%%%%%%%%%%%%%
\begin{frame}
	\frametitle{Bericht des Kassiers (1/3)}
	\begin{itemize}
		\item Einnahmen \textasciitilde 2kEUR/Monat (Housing)
		\item Ausgaben \textasciitilde 1kEUR/Monat (Strom)
		\item sowie RIPE, ISPA, Uplink, Steuerberatung
		\item Fallweise Investitionen: Hardware, Dark Fiber
	\end{itemize}
\end{frame}



%%%%%%%%%%%%%%%%%%%%%%%%%%%%%%%%%%%%%%%%%%%%%%%%%%%%%%%%%%%%%%%%%%%%%%%%
\begin{frame}
	\frametitle{Bericht des Kassiers (2/3)}
	Fokussierung an den Stockholdern:
	\begin{itemize}
		\item Lieferanten: Rechnungen per Bankeinzug bezahlen
		\item Housing: Kontenabstimmung mit Unterstützer*innen
		\item Steuerberater: Vorkontierung der Buchungen für Import
	\end{itemize}
\end{frame}



%%%%%%%%%%%%%%%%%%%%%%%%%%%%%%%%%%%%%%%%%%%%%%%%%%%%%%%%%%%%%%%%%%%%%%%%
\begin{frame}
	\frametitle{Bericht des Kassiers (3/3)}
	Fokussierung an den Stockholdern:
	\begin{itemize}
		\item jährlicher Stromanbieterwechsel $\rightarrow$ 4kEUR Ersparnis
		\item Kontoumstellung auf ein Konto 
		\item Laufende Kontrolle der offenen Rechnungen
		\item Keine Overhead-Tätigkeiten wie Kostenrechnung, Erlösrechnung, Zahlungsplan oder Forecastrechnung
	\end{itemize}
\end{frame}



%%%%%%%%%%%%%%%%%%%%%%%%%%%%%%%%%%%%%%%%%%%%%%%%%%%%%%%%%%%%%%%%%%%%%%%%
\begin{frame}
	\frametitle{Bericht der Rechnungsprüfer}
\end{frame}



%%%%%%%%%%%%%%%%%%%%%%%%%%%%%%%%%%%%%%%%%%%%%%%%%%%%%%%%%%%%%%%%%%%%%%%%
\begin{frame}
	\frametitle{Abstimmung über Entlastung des Vorstands}
	(für den Anteil der Funktionsperiode im Geschäftsjahr)
\end{frame}
\note[itemize] {
	\item Das Geschäftsjahr des Vereins beginnt (generell) unabhängig und
	(momentan) nicht gleichzeitig mit der Funktionsperiode des Vorstands
	\item Eine Angleichung wurde bereits vorgeschlagen, aber abgelehnt
	\item \url{https://wiki.funkfeuer.at/wiki/Regionen/Wien/Verein/201805_GV_Protokoll\#Antrag_auf_Angleichung_des_Gesch.C3.A4ftsjahres}
}



%%%%%%%%%%%%%%%%%%%%%%%%%%%%%%%%%%%%%%%%%%%%%%%%%%%%%%%%%%%%%%%%%%%%%%%%
\begin{frame}
	\frametitle{Wahl eines neuen Vorstands}
	\begin{itemize}
		\item Kandidaturen gemäß Wiki: \url{https://wiki.funkfeuer.at/wiki/Regionen/Wien/Verein/201905_GV\#Kandidaturen}
		\item Freiwillige*r Wahlleiter*in gesucht
		\item Ende der Wahlkartenausgabe
		\item Vorstellungsrunde der Kandidierenden
		\item Wahl der Vorstandsfunktionen
		\item Annahme der Wahl
	\end{itemize}
\end{frame}



%%%%%%%%%%%%%%%%%%%%%%%%%%%%%%%%%%%%%%%%%%%%%%%%%%%%%%%%%%%%%%%%%%%%%%%%
\begin{frame}
	\frametitle{Anträge}
	\begin{itemize}
		\item Änderung der Statuten
		\item ,,Mitgliedersituation'' (Christian Pock)
	\end{itemize}
\end{frame}



%%%%%%%%%%%%%%%%%%%%%%%%%%%%%%%%%%%%%%%%%%%%%%%%%%%%%%%%%%%%%%%%%%%%%%%%
\begin{frame}
	\frametitle{Antrag: ,,Änderung der Statuten''}
	\url{https://gitlab.com/funkfeuer/Statuten-Wien/merge_requests}
\end{frame}



%%%%%%%%%%%%%%%%%%%%%%%%%%%%%%%%%%%%%%%%%%%%%%%%%%%%%%%%%%%%%%%%%%%%%%%%
\begin{frame}
	\frametitle{Antrag: ,,Mitgliedersituation''}
	Antragsteller: Christian POCK
	% per Mail vom 22.05.2019
	\small{
	\begin{itemize}
		\item wie viele ordentliche Mitglieder und
		\item wie viele außerordentliche Mitglieder hat der Verein?
		\item wie/wann scheiden ordentliche Mitglieder aus, wenn sie mangels veralteter Kontaktdaten nicht mehr erreichbar sind UND sich seit Jahren nicht mehr melden/blicken lassen?
		\item wie viele Beitritte gab es pro Jahr die letzten 5 Jahre?
	\end{itemize}
	}
\end{frame}
\note[itemize] {
	\item Statuten §6 ,,Beendigung der Mitgliedschaft''
}



%%%%%%%%%%%%%%%%%%%%%%%%%%%%%%%%%%%%%%%%%%%%%%%%%%%%%%%%%%%%%%%%%%%%%%%%
\begin{frame}
	\frametitle{Allfälliges}
	\begin{itemize}
		\item ,,Neuhousing''-Projekt (Vortrag N. N., 10 min)
	\end{itemize}
\end{frame}



%%%%%%%%%%%%%%%%%%%%%%%%%%%%%%%%%%%%%%%%%%%%%%%%%%%%%%%%%%%%%%%%%%%%%%%%
\begin{frame}
	\frametitle{,,Neuhousing'' (1/2)}
	\begin{itemize}
		\item Herbst 2018: ,,Krypta muss absiedeln''
		\item Problem: Housing = Einnahmen, RIPE, Vereinsservices \& Lager
		\item Suche nach Keller und Glas
		\item Gegenseitige Abhängigkeiten in der Planung
	\end{itemize}
\end{frame}



%%%%%%%%%%%%%%%%%%%%%%%%%%%%%%%%%%%%%%%%%%%%%%%%%%%%%%%%%%%%%%%%%%%%%%%%
\begin{frame}
	\frametitle{,,Neuhousing'' (2/2)}
	\begin{itemize}
		\item Aktueller Plan: Metalab-Keller mit Dark Fiber der Wien Energie
		\item Ca. 10 kEUR DF-Erschließung, 10 kEUR Ausbau der Räume
		\item DF-Miete 200, Raum 150 EUR p.m.
		\item Alternativ: Keller des Volkskundemuseums
		\item DF-Miete 250 EUR p.m., Raummiete ,,hoffentlich niedrig''
	\end{itemize}
\end{frame}
\note[itemize] {
	\item Angebot Metalab-DF kostet insgesamt 36 kEUR, davon 12 Erstellung, Rest Nutzung
	\item Angebot VKM-DF kostet insgesamt 72 kEUR, davon 42 Erstellung
}



%%%%%%%%%%%%%%%%%%%%%%%%%%%%%%%%%%%%%%%%%%%%%%%%%%%%%%%%%%%%%%%%%%%%%%%%
\begin{frame}
	\frametitle{Das war's}
	Danke für's Dabeisein!
\end{frame}

%%%%%%%%%%%%%%%%%%%%%%%%%%%%%%%%%%%%%%%%%%%%%%%%%%%%%%%%%%%%%%%%%%%%%%%%
\end{document}
